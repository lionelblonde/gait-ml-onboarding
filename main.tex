\documentclass[12pt,a4paper]{article}

% Packages
\usepackage[utf8]{inputenc}
\usepackage{graphicx}   % For including images
\usepackage{amsmath}    % For mathematical equations
\usepackage{amssymb}    % For mathematical symbols
\usepackage{hyperref}   % For hyperlinks in the document
\usepackage{geometry}   % For setting page margins
\geometry{margin=1in}
\usepackage{cite}       % For managing citations
\usepackage{comment}    % For comment blocks

% Document Information
\title{\textsc{Gait} Project $n+1$}
\author{Lionel Blond\'e}
% \date{\today}

\begin{document}

% Title Page
\maketitle

% Table of Contents
\tableofcontents

% Introduction
\section{Introduction}
\label{s:intro}

The objective of this document
is to lay out the various options in have in addressing what we set out to tackle
in this $n+1$ iteration of the \textsc{Gait} project.

To have a detailed account of \textit{what} the project sets out to tackle, please refer to the
proposal, which goes into the motivations and plans in depth.
In here, we lay out the variety of options we have to address the raised research questions.
This exhibition entails general approaches and methods, but also elaborates on the technical
solutions.

The objective is to have a clear picture not only of the research landscape in the area
but also to surface the conceptual and technical challenges that are inherent to the area.
While the \textit{area} in question might have been phrased as ``gait modeling'' in previous
iterations, in this $n+1$ iteration, we formulate is as
\emph{learning locomotion controllers}.

The task of learning how a control policy that knows how to walk \textit{when embodied physically
in the real world} is challenging for a multiplicity of reasons
\cite{Radosavovic2024-dm, Caluwaerts2023-ko, Yang2023-jr, Kaufmann2023-yc, Smith2022-sm}.

We do not aim for embodied deployment.
What we want however is to learn controllers that are able to solve complex locomotion tasks
\textit{in silico}, \textit{i.e.}~in simulation.
While we wish to make use of such learned behaviors to derive insights and guide decisions in real
world tasks (such as surgical assistance), we do not aim to embody these in live robots.

An important point to make however is that most if not all the works in
\cite{Radosavovic2024-dm, Caluwaerts2023-ko, Yang2023-jr, Kaufmann2023-yc, Smith2022-sm}
make use of a simulated world in order to devise the controllers that they deploy in the real
world. In that sense, the research progress carried out during this project could therefore be
transferable and readily extended to research endeavors targeted at embodying gait controllers
in hardware. The difficulty in the deploying such behavior is in closing the gap between the
simulated and real world: the ``sim-to-real'' gap. Despite not having to deal with this gap due
to the lack of deployment need \text{per se}, we face the gap when we have to deal with patient
data. Indeed, data recorded by physicians (\textit{e.g.} motion capture markers) do not live in
the same space as the locomotion simulators at our disposal. Using such data to learn controllers
in simulation required an alignment system that either
\textit{(i)} embeds the markers in the simulation space, or
\textit{(ii)} morphs the simulation space to the world the markers live in.
The latter corresponds to the real world.
Going for option \textit{(ii)} was the goal of iteration $n$ of the \textsc{Gait} project.
This alignment of the simulator with the patient's morphology was lead by the \textsc{BioRob}
team at EPFL. The actual technical solution changed over the course of the project as team
members were coming and going. In the end, the simulation suite that allowed for the flexibily
needed to close the gap between patient data and \textit{in silico} world was
OpenSim \cite{Seth2011-ru}. It stands as a viable option.

Going back a few steps, consider the task of learning a locomotion controller in simulation.
In broad lines, the task requires the researcher practitioner to
\textit{(a)} choose or design a \emph{locomotion simulator}
\textit{(b)} choose or design a \emph{learning signal}
\textit{(c)} choose or design a \emph{learning approach}
\textit{(d)} choose or design a \emph{learning algorithm}
These axes or choice or design are themselves multi-faceted, and this document will attempt to
give the account of the options we have for each.
Note, ``choose'' refers to the selection and use of an already existing solution,
while ``design'' refers to the deliberate creation of a new solution.
As it stands, such a system is therefore highly modular.
Progress can be made on any combination of the four axes above.
How these axes are formulated is flawed in the sense that these axes are not independent from
each other. For example, the approach chosen determines to an extent the classes of algorithms,
or trait thereof, that one might opt for.
Still, it provides a salient language for us to navigate how to tackle the many-sided task. 

Due to the interactions between these axes, the remainder will not catalog the four axes one by
one, giving suggestions for each independently. Instead, we will consider the various learning
approaches that can be considered to solve the task, and expand from there as we go with options.
Organizing the exhibition as such hopefully gives a better account of those options compared to
a catalog with bi-directional links all over the place.

\section{Landscape}
\label{s:landscape}

By \textit{simulator}, we refer to a piece of software that our learning agent interact with in
the course of its learning process. A simulator is what facilitates a \emph{simulation}, which
Sheldon M. Ross defines in \cite{Ross2022-ra} as the imitation of the operation of a real-world
process or system over time. In our case, the process we are interested in is the human gait.
Formally, the simulator is modelled as a Markov Decision Process or MDP \cite{Puterman1994-pf}.
The latter consists of a tuple containing a state space (the sitations the agent finds itself in),
an action space (the decisions the agent might make), a dynamics model that governs the physics of
the world in which the agent interacts, and a reward process. MDPs are used in the formalism of
problems defined under the reward hypothesis, which sees each task as the maximization of a signal
called reward \cite{Bowling2022-li, Silver2021-uj}. Under this umbrella of problem formulations
lies reinforcement learning, \textit{abbrv.}~RL \cite{Sutton1998-ow}. It is used to
formalize problems of sequential decision making under uncertainty, and fits the problem of
locomotion control well if we see the gait problem as consisting in making a series of reactive
decisions consecutively in the interactive environment that is our simulator. We see immediately
that the RL agent is limited by the speed at which the simulator responds, which is why this
aspect is so critical. The choice of simulator will depend on the trade-off we want to strike.
Be it a prioritization of fidelity or of speed or execution, this choice does not change the task.
For the task to be fully defined however, the agent needs a reward or cost response upon
interaction with the system. This value should serve as guidance, informing the agent seeking
to maximize its cumulative reward ---or equivalently minimize its cumulative cost--- about how
well it is progressing toward the completion of the task.
This signal is designed \cite{Singh2009-cm, Randlov1998-qo, Russell1998-yk, Sims1994-zb},
and can very easily misshapen, leading to unforeseen effects or even exploits
\cite{Amodei2016-vg, Everitt2017-ql, Pan2022-ja, Langosco2022-eo, Skalse2022-xe, Sims1994-zb}
It is natural to think about energy parsimony when it comes to the primary signal optimized by
human when walking or running. This has been proved against in the case of horses in the past
however \cite{Hoyt1981-ma, Farley1991-fd}.
\footnote{Particularly interesting quote from \cite{Sims1994-zb}:
``It is important that the physical simulation be reasonably accurate when optimizing
for creatures that can move within it. Any bugs that allow energy leaks from non-conservation,
or even round-off errors, will inevitably be discovered and exploited by the evolving
creatures. Although this can be a lazy and often amusing approach for debugging a physical
modeling system, it is not necessarily the most practical.''}
Nevertheless, it has recently been found that minimizing energy consumption leads
to the emergence of gaits in legged robots \cite{Fu2021-hk}.
A very recent study documents that how agents can overfit to certain reward designs,
and emphasizes that one should construct a reward while keeping in mind that it is the 
reward accumulation that we want the agent to optimize toward \cite{Booth2023-ua}.

Simulators designed for control and locomotion are in general released with pseudo-benchmarked
reward function. These are the result of tedious hand-engineering, often tainted by the issues
raised and studied in \cite{Booth2023-ua}. They are usually aligned with the forward progress
of the agent in the sagittal plane \cite{Schulman2015-jt}. Another way to design a reward that
would incentivize the learning agent to adopt the desired behavior is to leverage the availability
of an expert at the task. This expert could be human or artificial, \textit{e.g.} a hard-coded
controller. This setting is called imitation learning, or learning from demonstration
\cite{Billard2008-jb, Atkeson1997-db, Bagnell2015-ni}.
Assuming alignment of expert and agent gaits, a sensible metric to use as cost could be the
average displacement between the reference expert gait.
Such a cost would be zero when the agent perfectly overlaps with the expert behavior.
The alignment assumption however becomes however increasingly in valid as we stray from purely
synthetic scenarios.

The locomotion controller we want to learn, denoted as $\pi$, maps the state space $\mathcal{S}$
onto the action space $\mathcal{A}$.
If demonstrations are state-action pairs, the problem of learning such a $\pi$ reduces to solving
a supervised learning problem (dubbed \textit{behavioral cloning}).

Note, learning a model with behavioral cloning (BC)
\cite{Pomerleau1989-nh, Pomerleau1990-lm, Ratliff2007-fc}
can be done \textit{offline}, \textit{i.e.}
without interacting with the simulator ---characterizing \textit{online} systems. BC is flawed
in more than one way \cite{Ross2010-eb, Ross2011-dn}.
To learn a viable policy, it requires diverse and action-annotated demonstrations in abundance.
Since this requirement is seldom met, practitioners often turn to apprenticeship learning (AL),
working in lower data regimes, and at least equally importantly, not requiring the expert data
to be annotated with controls (actions).
Indeed, the controls are rarely readily available (\textit{e.g.} video demonstrations).
AL \cite{Abbeel2004-rb, Ratliff2006-kj, Kolter2008-mr} is traditionally consisting
of two stages: 
\textit{(i)} inverse reinforcement learning (IRL; to derive a cost from the expert behavior), and
\textit{(ii)} reinforcement learning (RL; to derive a behavior from the cost being learned).
These stages are sometimes done once, one after the other, or in alternation throughout the entire
learning process \cite{Ho2016-xn}.

Trying to recover the \textit{real} reward (the one optimized by the expert and that led to its
behavior) has been shown to be an ill-posed problem \cite{Ziebart2008-fe}.
A relaxation of AL consisting in learning only a \textit{surrogate} cost has proved enough,
and has led to state-of-the-art results with
generative adversarial imitation learning (GAIL) \cite{Ho2016-bv}.
Being a relaxation of GAIL, it can learned from control-annotated demonstrations, or from
observations (states) alone, without actions from the expert.
The first setting is called ``ILfD'', the second ``ILfO'', for ``Imitation Learning from
Observations'' \cite{Merel2017-lo, Liu2017-dr, Reed2018-ga, Fu2018-zu, Aytar2018-by,
Torabi2018-fg, Torabi2018-nb}.
Adversarial imitation learning is the go-to way to learn a sensible cost (reward) signal,
but what about step \textit{(ii)} above ---the RL method to use in order to learn
a controller from the surrogate reward?

Prime candidates to solve the RL outer loop and that have proved successful in various scenarios
including locomotion tasks are \emph{trust-region methods}.
The figure-heads of this family of algorithms
are TRPO \cite{Schulman2015-jt}, PPO \cite{Schulman2017-ou},
MPO \cite{Abdolmaleki2018-sp}, V-MPO \cite{Francis_Song2020-sd},
and MDPO \cite{Tomar2020-ts}.

The latter, which stands for ``Mirror Descent Policy Optimization'' has had its first surge of use
recently in an LLM context \cite{Gunter2024-hu}.
MDPO is, according to the original paper, simpler and more stable than PPO, which is infamously
tedious to tame \cite{Engstrom2019-jt, Huang2022-bv, Moalla2024-xq}.
MDPO has been introduced in two variants, one stems from PPO,
the other from SAC \cite{Haarnoja2018-bm, Haarnoja2018-uo},
a method that falls under an architecture called \textit{actor-critic}
\cite{Crites1995-hn, Konda2000-ef, Sutton2001-vh}.
While PPO is \textit{on-policy}, SAC (which stands for ``Soft Actor-Critic''),
is \textit{off-policy}.
A method is said to be on-policy when the policy being learned and the one being used are one and
the same. When they are distinct, the method is off-policy (and the two policies are called the
\textit{behavior} and the \textit{target} policies).
We will shortly tackle why off-policy-ness is of great interest when dealing with real-world RL
in scenarios where safety is critical and where data scarcity is being faced
(\textit{cf.}~\ref{ss:efficiency}).

If one has access to demonstrations in abundance \emph{and} organized in complete sequences 
(\textit{i.e.} not sub-sampled)
then one can turn to \emph{sequence modeling}
(\textit{cf.}~\cite{Goodfellow2016-ev}, \textsc{Chap.}~10).
Tackling control problem as sequence prediction is not new, but is disregards the fact that
control agents, as interactive in nature, must comply with the physics of their environment
\cite{Salzmann2020-jl}. Typical technical solutions have involved RNN/LSTM backbones for
their ability to propagate information in time across timesteps.
It does so by compressing previous information seen in the trajectory (accumulating over time)
into a fixed size hidden state.

Mainstream attention has moved onto a different architecture called
the Transformer \cite{Vaswani2017-lk},
which splits sequences in tokens from a fixed vocabulary,
and learning how to build sequences by trying to predict the next token given all the previous
ones in the sequence.
The core mechanism at play is \emph{attention}, and specifically \emph{causal attention}, where
the future is masked: the model can not attend to future token in its prediction.
This ensures temporal consistency.
The two concurrent works that first tackled control problems
\textit{(i)} as sequence modeling problems,
\textit{(ii)} with transformers, are
\cite{Chen2021-qq} and \cite{Janner2021-wn}.
An vast array of works has spawned from these.
Still, they are inherently flawed because they are not trained online, but from an offline dataset.
Despite this flaw, they have proved very effective for real-world humanoid locomotion tasks
\cite{Radosavovic2024-xl}
and real-world robotics tasks \cite{Brohan2022-vs, Brohan2023-rm}.
The obvious caveat is that it is a challenging engineering endeavor.
Hybrid solutions have led to the best results so far.
These leverage the power of transformers as feature extractors from sequences, but the task is of
the RL type, \textit{i.e.} maximize the cumulative reward, or IL type.
Humanoid locomotion via RL is the goal of \cite{Radosavovic2024-dm},
with broader robotics tasks via RL being the focus in the Q-Transformer \cite{Chebotar2023-hq},
and via BC (IL) in Behavior Transformer (BeT) \cite{Shafiullah2022-ux}.
Diffusion heads have also been used on top of transformer backbones with great practical success,
like in the Diffusion Policy \cite{Chi2023-is}.
Diffusion has also been successful for locomotion:
Motion Diffusion Model (MDM) \cite{Tevet2023-pv}.

\section{Desiderata}
\label{s:desiderata}

Ultimately what we want is to learn high-performance \footnote{Note, the intricacies of the
performance objective depend on the specific task formulation at hand.}
\textit{in silico} behavior in the shortest
time possible, where the \textit{time} is question encompasses
\textit{(i)} the wall-clock time it takes for the learning agent to solve the task, and
\textit{(ii)} the number of man-hours it takes for a given research idea to be validated
or invalidated.
Of course, those two are related: the shorter an experiment takes, the sooner the practitioner can
iterate on the obtained results and run the next experiments with the adapted code.
But those are two different axes on which the practitioner can make progress for the pipeline to
be more efficient in time.
To have the lowest time possible, our system should involve a simulator that is as fast as possible
and a learning algorithm that requires the fewest possible number of interaction with the simulator
to learn the optimal behavior for the task.

We will treat those in reverse order, first talking
about the latter (a desideratum called \textit{sample efficiency}) in \ref{ss:efficiency},
before dealing with the former (the speed of execution) in \ref{ss:speed}.

\subsection{Sample Efficiency}
\label{ss:efficiency}

Since simulators can be complex and therefore slow,
it is of primary interest to treat with methods
that require few interactions with the simulator,
\textit{i.e.} sample-efficient methods.

There are two main avenues for potential major improvements in sample efficiency:
off-policy learning (described briefly earlier in \ref{s:landscape}),
and model-based learning.
Learning off-policy allows for substantial sample efficiency gains because it is compatible
with experience replay. This mechanism consists in keeping a FIFO-queue buffer containing the
$N$ most recent transitions, and training the agent's policy (and value) on its contents.
It is \textit{off}-policy because the sampling distribution corresponding to getting transitions
uniformly from the replay buffer is a mixture of past policy updates, which is (in non-degenerate
cases) different from the policy being learned. Those methods are usually less friendly, but the
sample-efficiency gains are arguably well worth it in practical scenarios.

Model-based learning consists in learning a model of the environment in addition to learning
a policy (and value). It can be abbreviated MBRL for ``model-based RL''. What is modeled has
traditionally been the elements of the MDP, \textit{i.e.}
the forward dynamics $f: (s_t, a_t) \mapsto s_{t+1}$,
the reward process $r: (s_t, a_t) \mapsto r_t$,
and the termination flag (whether the episode is over or not) $d: (s_t, a_t) \mapsto d_t$.
This is because such a model can then act as a substitute for the real environment.
In that case, the interaction needs are lowered, and the method is more sample efficient
(technically). This category of MBRL approaches (replacing the environment with a learned one)
was introduced by Sutton with Dyna \cite{Sutton1991-cp}.
As such, we will use the adjective ``Dyna-style'' to refer to this family of methods.
MVE \cite{Feinberg2018-tv} extends Dyna, and
STEVE \cite{Buckman2018-jd} extends MVE.
In the same vein, MBPO \cite{Janner2019-sk} is closely related to MVE and STEVE.

In addition to Dyna-style methods, there are ``SVG-style'' methods, where SVG refers to the
``Stochastic Value Gradients'' work \cite{Heess2015-va}.
SVG builds on ``Value Gradients'' \cite{Fairbank2012-rp},
and has been revisited more recently in \cite{Amos2021-wd}.
The origin of this approach could be traced back to the appearance of
back-propagation-through-time (BPTT)
\cite{Rumelhart1986-ls, Robinson1987-px, Nguyen1990-zx, Werbos1990-qa,
Williams1990-xw, Jordan1992-wn, Grzeszczuk1998-ij}
It has then been continued in more modern setting by \cite{Schmidhuber2011-mt, Deisenroth2011-ya}, 
and very recently in the very successful DreamerV\{1-3\} methods
\cite{Hafner2019-oa, Hafner2021-td, Hafner2023-wk, Lin2023-ql}.
These methods learn a model of the world with neural networks so that the learned approximation of
the world can be part of the episode-spanning computational graph created by the RL objective.
Effectively, the gradient propagate backward through the trajectories, from the termination to
the initialization.
For how these methods operate, the family is sometimes said to use ``pathwise derivatives'',
to backpropagate through paths.
While SVG uses only paths built from real samples (actually collected in rollouts in simulation),
the Model-Augmented Actor-Critic (MAAC) \cite{Clavera2020-ha}
uses both real rollouts and model rollouts (generated from the learned model of the environment).
Another method from that family is SuperTrack \cite{Fussell2021-uf}.
The policy is not optimized via RL however, but with IL.

The third approach to MBRL consists in learning a model of the world and deriving a
policy by \emph{planning} on the model with a method like MPC (model-predictive control).
MPC is an online optimal control method
\cite{Bryson1969-mg, Bertsekas2000-yi, Kirk2004-dq}.
By virtue of being online, it is adaptable by design.
It might look closer to optimal control than RL but more modern
MPC-style methods use a learned \textit{termination value or critic}
for the learned controller to always be able to gauge the long-term of its plans, as in
TD-MPC2 \cite{Hansen2024-ld}.

MPC is particularly appealing when the uncertainty of the model of the world is modeled,
since the policy can then make good use of this information,
as in PETS \cite{Chua2018-fn} and POPLIN \cite{Wang2020-jd}.

For the three ways of conducting MBRL (Dyna-style, SVG-style, MPC-style),
it is common to learn models that do not follow the strict signatures of the MDP elements.
Instead, a more complex model is often learned, the forward model is derived from it if needed.
More complex models allow for more salient and holistic neural representations to be learned,
leading to better values and controller down the line. They are called \emph{world models},
which is an appropriately generic name considering how one might differ from one work to the next.
Some references to world models include
\cite{Ha2018-sa, Hafner2018-zm, Hessel2021-ah, Micheli2023-mr, Ding2024-oq, Alonso2024-jo}.
So:
\begin{description}
    \item[Dyna-style]
        The Dyna-style methods learn a model of the world to use this as an approximation
        instead of interacting with the simulator, hence \emph{technically} reducing how many times
        the agent must interact with the simulation to learn.
    \item[SVG-style]
        The SVG-style methods learn a \emph{neural} model of the world to use in the objective's
        computational graph and be part of the optimization problem directly,
        in the hope that leveraging world gradients will make the controller learn faster.
    \item[MPC-style]
        The MPC-style methods learn a model of the world to plan on, and planning is generally
        less costly in data points than model-free RL methods to derive a policy.
\end{description}

All three methods have their respective merits.
The most \textit{neural land} of the three is the SVG-style family.
It has had successes (see above) but mostly disappoints for the expected successes
it should have based on how paradigm-shifting end-to-end neural pipelines have been in essentially
all other sub-domains of machine learning.
The culprit is the behavior of the gradients of the world models in control scenarios
(let us call those \textit{world gradients}).
DreamerV\{1-3\} (SVG-style) has been extremely successful in \emph{discrete} control,
but TD-MPC2 (MPC-style) still yields better results in \emph{continuous} control.
This is likely due to the world gradients being tedious to deal with in continuous control tasks.
Backpropagation of annoying gradients (too big or too small) has been a hot topic for as long
as neural networks have existed ---see previous references cited above on BPTT.

A way to potentially get better world gradient is to use a \emph{differentiable simulator}.
We will abbreviate those as \textit{diffsims}.
Note that in such a case, there is no world model \textit{a priori}.
Instead of designing a learning task for a world model to extract useful and salient information
from the world by itself ---learning just what it \emph{needs} to solve the task,
the world gradients the agent gets are derived from hard-coded operations encoded via
deep learning frameworks.
As such, diffsims are a way to benefit from the sample efficiency of SVG-style model-based methods
without the need to learn anything else than the agent's policy (and value).
They are not \textit{learned from data} (note, \textit{where} this data comes from is a hurdle
of its own in world modeling) but are \textit{crafted} by the seasoned practitioner
who injects domain knowledge and well-vetted structural biases in it.
The diffsim designer has knowledge about the domain and knows how the design choices impact
the interaction time (direct consequence of the integration from dynamics to kinematics).
The designer can strike the desired trade-offs between fidelity with the real physics and speed.
More sophisticated models can lead to higher fidelity, but are more costly to integrate over.

Works that involve pathwise derivatives in a diffsim include
SHAC \cite{Xu2022-bz}, AHAC \cite{Georgiev2024-rs}, DiffMimic \cite{Ren2023-yc},
PODS \cite{Mora2021-pp}, and ILD \cite{Chen2022-zd}.
While all these are optimized with an RL objective, ILD is optimized with an IL one.
Also, PODS requires second order information (about the Hessian) from the diffsim.

Plus, diffsims are hard-coded and therefore less prone to \textit{model hacking} (also refered
to as \textit{model exploitation} \cite{Ross2012-yj}
or \textit{model bias} in PILCO \cite{Deisenroth2011-ya}).
This phenomenon is observed when the policy uses the model of the world in ways that were
unintended by the practitioner. It \textit{hacks} or \text{exploits} the model to solve the task
it was asked to solve, but did it in unforeseen ways at test time.
Model bias is mentioned in SVG \cite{Heess2015-va}, ME-TRPO \cite{Kurutach2018-lw},
PIPPS \cite{Parmas2018-se}, and the authors of MBPO \cite{Janner2019-sk} write that
``errors in model can undermine the advantage from model-based data augmentation''.
This is typical when dealing with black boxes like neural nets.
The phenomenon is close in spirit to reward hacking.
Note, model bias is a hurdle for all three families of model-based methods.
It is easier to correct in diffsims simply because, since the simulator is hard-coded and readable,
we can \textit{know} what was exploited. This part of the pipeline is a white box.
Model exploitation loopholes are especially critical to avoid in SVG-style methods since
feedback propagate along trajectories and pollutes everytime upstream if erroneous.
Also, locomotion simulators are tedious to turn into diffsims notably because of the
\emph{contacts}, which introduce discontinuities in the simulator's response, and therefore
Dirac deltas in the diffsim gradients.
Design choices with respect to contact design are therefore particularly critical.
Harder contacts are better for fidelity and therefore closing the sim-to-real gap,
but softer contacts (dampened, surfaces like ground being permeable) remove the risk of introducing
spikes in world gradients.
Another point of design: in order to back-propagate from the RL objective through the diffsim
backward through long trajectories, the reward itself need be a differentiable signal
with respect to the quantities of interest, which is not necessarily obvious.
In the case of IL however, the displacement error need only the dynamics to be differentiable.
Lastly, since diffsims are tedious to engineer, using them is often an exercise in patience.

Some relevant references on diffsims:
\cite{Degrave2016-vr, Hu2019-ax, Falisse2019-ge, Hu2020-eb, Qiao2020-um, Gradu2020-kq,
Geilinger2020-ar, Mora2021-pp, Heiden2021-fb, Clarke2021-rr, Werling2021-da, Qiao2021-zn,
Huang2021-le, Heiden2021-ey, Grinsztajn2021-mr, Daniel_Freeman2021-az, Lin2022-jr, Xu2022-bz,
Suh2022-mn, Allen2022-vc, Chen2022-zd, Ren2023-yc, Georgiev2024-rs}.
These range from the introduction of new simulators, to analyses of gradients types
(zero-th versus first order gradients), to new ways to leverage diffsims to a fuller extent.

\subsection{Execution Speed}
\label{ss:speed}

This section is meant to contain items justifying the need for high execution speed if one wants
to close the sim-to-real gap in a reasonable amount of time (for RL standards).

In 2019, the OpenAI robotics team ---now defunct--- leveraged a techique that led
them to train a dextrous robotic hand \footnote{The ``Shadow Dexterous Hand'', from
\url{https://www.shadowrobot.com/dexterous-hand-series/}}
to solve the Rubik's cube puzzle \cite{OpenAI2019-vy}.
The endeavor consisted in first \textit{solving} the dextrous in-hand manipulation of the cube,
which was reported in \cite{OpenAI2018-sm}, before integrating an off-the-shelf Rubik's cube
solver into the system (engineering feat on its own).
In order to solve the dextrous in-hand manipulation task, the engineers at OpenAI used a
technique called \emph{domain randomization} (DR), originating in works such as
\cite{Tobin2017-ir, Tan2018-ux}.
The technique is akin to the data augmentations carried out in self-supervised learning
but these augmentations are done at the rendering engine level. The engineers developed a
system that allows fast and customizable rendering of robotics environments. They called the
system the OpenAI Remote Rendering Backend (ORRB) \cite{Chociej2019-ot}.
In short, the system allows for the randomization of various aspects of the rendered robotic
hand and objects in simulation (variety of textures, colors, etc.).
Domain randomization very useful to be able to communicate to the agent what is relevant and what
is not. By seeing a variety of textures and colors that do not change the task in nature, we
want the model to understand what it should be \emph{invariant} to.
From this diversity comes robustness with respect to the unknowns of the real world.

So, if one want to close the gap, showing lots of things to the agent is crucial
--- \textit{e.g.} via DR, and for that to be feasible, we need fast simulation engine backends.

Performant rendering engine have proved crucial since they allow techniques like DR.
The video game industry has invested huge amounts to have performant rendering backends.
It is therefore not surprising to witness the use of game engines to train agents via control
approaches.
The problem of solving locomotion problems in simulation with control algorithms
has not only been tackled by the
robotics or biorobotics crowds, but also by the \emph{character animation} crowd,
originiting in works such as \cite{Raibert1991-wu, Van_de_Panne1993-is, Grzeszczuk1998-ij}.
An account of ``pre-2012'' state-of-the-art works carried out in
interactive character animation using simulated physics
is given in \cite{Geijtenbeek2012-mq}.
Later works include: MACE \cite{Peng2016-fs},
DeepLoco \cite{Peng2017-fg},
Phase-Functioned Neural Network (PFNN) \cite{Holden2017-wv},
DeepMimic \cite{Peng2018-ap},
UniCon \cite{Wang2020-cz},
AMP \cite{Peng2021-ge},
SuperTrack \cite{Fussell2021-uf}
\footnote{Note, SuperTrack is from a game engine studio,
uses a world model, as well as pathwise derivatives.},
DeepPhase \cite{Starke2022-za},
AdaptNet \cite{Xu2023-ug}.
The goal of character animation is for the rendered video to be believable enough for the viewer,
and in some cases controllable ---desirable trait for interactive mediums like video games.
The goal is \emph{not} to be as strictly faithful to the real physics of the world as possible.
As such, the physics engine backends used in character animation are often simplifying how
contacts are handled. Indeed, as described earlier, contacts introduce discontinuities in the
simulator's response, which makes the integration of the dynamics into kinematics more numerically
unstable. When a foot hits or leaves the ground, there is a sudden change in the system's dynamics,
causing:
\textit{(a)} instability in numerical integration due to rapid force changes,
\textit{(b)} friction and slipping complications, which are hard to model accurately,
\textit{(c)} impulse forces during impact, requiring careful handling to avoid simulation errors,
\textit{(d)} constraint violations, such as feet penetrating the ground or sliding unexpectedly.
Dampeners can solve contact problems in locomotion simulators by smoothing the force transitions
during contact events. Engines usually leverage such techniques, unless fidelity with real world
locomotion is paramount.
MuJoCo \cite{Todorov2012-gc} is often considered superior to many other simulation engines
for handling contacts due to its advanced and efficient contact modeling.
MuJoCo uses a form of \emph{soft contact model} with damping to handle contacts
Specifically, MuJoCo employs a compliant contact model
\footnote{A \emph{compliant contact model} is a method of simulating contacts between objects by
allowing small deformations or penetrations at the contact point, instead of treating the objects
as perfectly rigid. This approach uses spring-like forces and dampers to model the interaction
between the contacting surfaces.}.
where contacts are modeled using a
combination of elastic (spring-like restoring force) and damping (energy-dissipation) forces.

The major hurdle of MuJoCo is that it was developed to be used on CPU hardware.
But recent works, including RP1M \cite{Zhao2024-vv},
have started using MuJoCo XLA (MJX)
\footnote{\url{https://mujoco.readthedocs.io/en/stable/mjx.html}},
which supports simulation in parallel with GPUs.
While very high-fidelity biorobotics-inspired simulators such as
OpenSim \cite{Seth2011-ru} and
MyoSuite \cite{Caggiano2022-ag}
will likely not be ported to GPU, chances are the control community will make the switch
from the original MuJoCo backends (CPU) to MJX (GPU).
This transition allows for the training of agents in a massively parallel fashion.
Nvidia's Isaac Gym \cite{Makoviychuk2021-wv} did report tremendous speed-ups due to the
parallelization enabled by having the GPU hardware as the core design choice.
Nvidia also developed \texttt{dflex}
\footnote{\url{https://github.com/NVlabs/DiffRL/tree/main/dflex}},
a PyTorch-based physics diffsim, which they used in
SHAC \cite{Xu2022-bz} and AHAC \cite{Georgiev2024-rs}.
For reasons unknown, \texttt{dflex} is not maintained anymore since 2022, despite being in use
in their 2024 paper (see appendix of the latter for details).

MJX was built by core contributors to both MuJoCo
and Brax \cite{Daniel_Freeman2021-az} (diffsim),
who will together continue to support both Brax and MJX
---according to the official MJX documentation.
So, the best candidate when it comes to diffsim seems to be Brax.
The only caveat is that it requires the use of JAX deep learning framework.
While there exists an official PyTorch wrapper in Brax, it does not allow gradients to flow
through, making it a non-differentiable GPU-based simulator.
Note, Brax takes a fair amount of shortcuts to make its primitives differentiable
(shared in the companion paper \cite{Daniel_Freeman2021-az}).

When it comes to \emph{musculo-skeletal} models, \texttt{dflex} has one.
Compared to the ones from 
OpenSim \cite{Seth2011-ru} and
MyoSuite \cite{Caggiano2022-ag}
which are CPU-based and therefore slow,
\texttt{dflex}'s is GPU-based (fast) \emph{and} differentiable.
The major problem being that \texttt{dflex} is not maintained anymore.
Note, a private conversation with one of the main authors of the AHAC paper
revealed that the musculo-skeletal model they have is available from head-to-toe
despite being only from hips-to-toe in the papers (SHAC and AHAC).
The upper body has simply been commented out in the codebase.
The author encouraged use to communicate with them about our project to that they can
assist us in using their \textit{next-gen} simulator engine, \texttt{warp}
\footnote{\url{https://github.com/NVIDIA/warp}}.
Nvidia left \texttt{dflex} development, and has yet to release its successor diffsim
based on \texttt{warp}. This day might never come.
Note, Isaac Gym development has also been abandonned, probably to promote the move to
\texttt{warp}.
To conclude, GPU-based simulators have allowed huge speed-ups in training and therefore
development time. The main player in the field is Nvidia ---not surprising considering their
place in the GPU market, but now is an awkward time where they seem to be switching backend.
Using MJX seems to be the smarted move support-wise.
To conclude:

\emph{Both MJX and Brax (diffsim) are active and led by Google, but are written in JAX.
Isaac Gym and \texttt{dflex} (diffsim) are inactive and led by Nvidia, but they are in PyTorch.}

\section{NeurIPS Competitions: Learning to Walk}
\label{s:l2w}

\subsection{From 2017 to 2024}

\subsubsection{NeurIPS 2017: Learning to Walk}

\begin{itemize}
    \item \textbf{Objective}: Train a bipedal agent to walk as far as possible in a simulated
        environment using reinforcement learning.
    \item \textbf{Challenges}: Realistic walking behaviors in a physics-based MuJoCo simulation.
    \item \textbf{Musculo-skeletal Models}: \textbf{No} – The simulation involved direct control
        of joints (torques) rather than muscles.
    \item \textbf{Outcome}: Highlighted the need for efficient policy learning in basic locomotion
        tasks.
\end{itemize}

\subsubsection{NeurIPS 2018: AI for Prosthetics}

\begin{itemize}
    \item \textbf{Objective}: Control a prosthetic leg to enable walking for a human model with a
        simulated amputation using \textbf{OpenSim}.
    \item \textbf{Challenges}: Human-prosthetic interaction, simulating biomechanics of both the
        prosthetic and remaining leg.
    \item \textbf{Musculo-skeletal Models}: \textbf{Yes} – The simulation used musculo-skeletal
        models with muscle-based controls, making it more realistic in mimicking human gait.
    \item \textbf{Outcome}: Showed the importance of biomechanical realism in prosthetic
        simulations.
\end{itemize}

\subsubsection{NeurIPS 2019: Learn to Move – Walk Around}

\begin{itemize}
    \item \textbf{Objective}: Agents had to walk and run in a complex 3D environment using
        \textbf{OpenSim}, with adaptive movement across various terrains.
    \item \textbf{Challenges}: Terrain complexity, requiring stability and adaptability.
    \item \textbf{Musculo-skeletal Models}: \textbf{Yes} – OpenSim's musculo-skeletal models with
        muscle-based controls were used, making the agent control more human-like.
    \item \textbf{Outcome}: Advanced the ability to generalize movement behaviors across different
        environments with more complex biomechanical models.
\end{itemize}

\subsubsection{NeurIPS 2020: Learn to Move – Adaptive Arm}

\begin{itemize}
    \item \textbf{Objective}: Focused on upper-limb control and object manipulation using a
        simulated arm.
    \item \textbf{Challenges}: Adapting to changing goals and constraints in fine motor control.
    \item \textbf{Musculo-skeletal Models}: \textbf{Yes} – The competition used musculo-skeletal
        modeling for the arm, controlling muscles rather than direct torques.
    \item \textbf{Outcome}: Expanded RL applications to include fine motor skills, emphasizing
        musculo-skeletal dynamics.
\end{itemize}

\subsubsection{NeurIPS 2021: The Walkers}

\begin{itemize}
    \item \textbf{Objective}: Control four-legged walkers across challenging terrains.
    \item \textbf{Challenges}: Handling uneven terrain, slopes, and moving obstacles.
    \item \textbf{Musculo-skeletal Models}: \textbf{No} – The simulation used joint-based controls
        (torques) rather than muscle-based controls.
    \item \textbf{Outcome}: Focused on the adaptability of RL agents to dynamic terrains and
        obstacles without introducing musculo-skeletal complexity.
\end{itemize}

\subsubsection{NeurIPS 2022: The Walkers – Humanoids}

\begin{itemize}
    \item \textbf{Objective}: Agents had to control humanoid walkers, adding complexity due to the
        high-dimensionality of the humanoid model.
    \item \textbf{Challenges}: Stability and balance of a humanoid across complex terrains.
    \item \textbf{Musculo-skeletal Models}: \textbf{No} – The controls were based on joint torques,
        not muscle-based controls.
    \item \textbf{Outcome}: Explored advanced techniques like \textit{hierarchical RL} and
        \textit{imitation learning} for humanoid locomotion, but without musculo-skeletal models.
\end{itemize}

\subsubsection{NeurIPS 2023: Robust Locomotion}

\begin{itemize}
    \item \textbf{Objective}: Develop controllers for robust locomotion across various terrains and
        dynamic conditions.
    \item \textbf{Challenges}: Emphasis on robust performance in unseen environments.
    \item \textbf{Musculo-skeletal Models}: \textbf{No} – The simulation focused on controlling
        joint torques rather than muscles.
    \item \textbf{Outcome}: Leveraged \textit{domain randomization} and
        \textit{adversarial training} to improve the generalization of locomotion controllers,
        but without muscle-based models.
\end{itemize}

\subsubsection{NeurIPS 2024: Physiological Dexterity and Agility in Bionic Humans}

\begin{itemize}
    \item \textbf{Objective}: This challenge, called the MyoChallenge\footnote{%
        \url{https://sites.google.com/view/myosuite/myochallenge/myochallenge-2024}},
        focuses on creating controllers for musculo-skeletal systems in bionic human models.
        The goal is to push the limits of physiological dexterity and agility by simulating fine
        motor control and robust, dynamic movements.
    \item \textbf{Challenges}: 
    \begin{itemize}
        \item Controlling complex musculo-skeletal systems, including simulating physiological
            responses.
        \item Achieving robust dexterity and agility in both upper- and lower-limb tasks, with a
            focus on dynamic and adaptive behaviors.
    \end{itemize}
    \item \textbf{Musculo-skeletal Models}: \textbf{Yes} – The challenge uses musculo-skeletal
        models, with muscle-based controls allowing for physiological accuracy in movement.
    \item \textbf{Outcome (Expected)}: This challenge will drive innovations in modeling
        human-like movement and control, as well as in developing advanced bionic systems that
        mimic human agility and dexterity.
\end{itemize}

\subsection{Summary \textit{vis-a-vis} Musculo-skeletal Models}
\begin{itemize}
    \item \textbf{Included in}: 2018, 2019, 2020 (focused on prosthetics, adaptive walking, and
        arm control, where simulating realistic muscle actions was crucial).
    \item \textbf{Excluded in}: 2017, 2021, 2022, 2023 (focused more on joint-based or
        torque control, particularly in robotic and humanoid walking challenges).
\end{itemize}

\section{Concretely}

I suggest the following (``H.'', ``J.'', and ``L.'' are Hugues, Joao, and Lionel respectively):
\begin{enumerate}
    \item H. presents his bioscience background in front of the whole team.
    \item H. presents a taylored presentation about biomechanical models in front of
        this project team only. The contents: Hill models, Raash models, Hunt-Crossley models,
        Featherstone models, generalized coordinates, etc.
    \item H. goes through all the "Learning to Walk" competitions listed out in \ref{s:l2w}
        and walk us through what the winning teams did to get the first place. Podium contestants
        with notable or out-of-the-ordinary approaches should be discussed well.
    \item H. should get familiarize himself with running RL and IL experiments in the 
        LocoMuJoCo \cite{Al-Hafez2023-dn} suite. At the same time, if timeline forces it, L./J.
        will be conducting experiments to learn controllers in musculo-skeletal simulators.
        If J. is actively working of diffsims without muscles, then L. will work on the
        muscle-actuated simulations.
    \item The role of L./J. is to assist H. as much as possible in being proficient in handling
        simulators and understanding RL/IL methods --this assistance is hands-on, since it will
        take time for H. to get accustomed to this new machinery. Still, someone at ease with
        Python and biomechanics will likely progress fast, especially with our continued
        assistance.
    \item Concerning \textit{which complex simulators}, we will use:
        \textit{(a)} MyoLeg, a MuJoCo model, used via the MyoSuite Python API, and
        \textit{(b)} Hyfydy models used via the SCONE Python API.
        Both are very fast, although not running on GPU. 
    \item Whether diffsim and/or GPU simulations will be used in this project depends entirely on
        the research interests and endeavors or H., J., and L. It is not really a item of primary
        interest for the project \textit{per se}.
    \item H. has some familiarity with PyTorch. L. and J. are proficienty in PyTorch.
        There have nevertheless evidence that JAX has been instrumental in real-world 
        RL use \emph{where PyTorch failed to deliver in terms of speed},
        notatly in \cite{Smith2022-sm}.
        L. will use JAX to revive a previous project that was suffering from long training times;
        maybe that might be also a good occasion for H. to learn Jax\footnote{%
        The models and simulators previously mentioned (Hyfydy, MuJoCo, etc.)
        are outside the computational graph, and therefore agnostic \textit{w.r.t.}
        deep learning automatic differentiation (AD) frameworks}.
    \item In parallel, L. will be in communication with Eric Heiden from the NVIDIA \texttt{warp}
        team to discuss various things related to their muscle-actuated locomotion diffsim.
\end{enumerate}

\section{Extras}
\label{s:best}

A very candidate for locomotion tasks seems to be TD-MPC2 \cite{Hansen2024-ld}
(the \emph{why} was discussed earlier).

A very promising environment: LocoMuJoCo \cite{Al-Hafez2023-dn},
which has domain randomization capabilities.

There might be interesting considerations in
``Temporal Cycle-Consistency Learning'' (TCC) \cite{Dwibedi2019-jv}.

Some references of \emph{motion capture} (MoCap):
\cite{Merel2017-lo, Merel2019-rn, Merel2020-ea, La_Barbera2021-kq, Bohez2022-jt,
Liu2022-lw, Winkler2022-nr, Sun2023-bl, Radosavovic2024-xl}

Comparison of the impact of four different action parameterizations (torques, muscle-activations,
target joint angles, and target jointangle velocities) in terms of learning time,
policy robustness, motion quality, and policy query rates: \cite{Peng2016-gc}.

Interesting tool to scale musculoskeletal models automatically, along with inverse kinematics and
inverse dynamics (based on Nimble): \cite{Werling2023-mg}.
The algorithm is published as an open source cloud service at \url{AddBiomechanics.org}.
Also under the topic of \textit{system/parameter identification},
\cite{Valente2014-vq} asks the question of whether patient-specific musculoskeletal models
are robust to uncertainties in its parameters.

% References
\bibliographystyle{alpha}
\bibliography{my_bib} % (without the .bib extension)

\end{document}
